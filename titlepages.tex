\renewcommand{\thefootnote}{\fnsymbol{footnote}}

%kortkommandon för mailaddresserna
\newcommand{\andsunds}{andsunds@student.chalmers.se}




\pagenumbering{roman} %%Romersk sidnumrering i början
\begin{titlepage}
\newgeometry{top=3cm, bottom=2cm}

\newcommand{\HRule}{\rule{\linewidth}{0.5mm}} % Defines a new command for the horizontal lines, change thickness here

\center % Center everything on the page
 
%------------------------------------------------------------------------------------
%	HEADING SECTIONS
%------------------------------------------------------------------------------------

\textsc{\huge Chalmers tekniska högskola}\\[1.5cm] % Name of university/college
\textsc{\Large Kurs}\\[0.2cm] % Major heading such as course name
\textsc{\large Moment }\\[0.5cm] % Minor heading such as course title

%------------------------------------------------------------------------------------
%	TITLE SECTION
%------------------------------------------------------------------------------------

\HRule \\[0.4cm]
{ \LARGE \bfseries 
Titel
}\\[0.4cm] % Title of  document
\HRule \\[1.5cm]
 
%------------------------------------------------------------------------------------
%	AUTHOR SECTION
%------------------------------------------------------------------------------------

\begin{minipage}{0.4\textwidth}
\begin{flushleft} \large
\emph{Författare:}\\
Andréas Sundström\footnotemark{}
\end{flushleft}
\end{minipage}
~
\begin{minipage}{0.4\textwidth}
\begin{flushright} \large
\emph{Lärare:} \\
N N
\end{flushright}
\end{minipage}\\[3cm]

\setcounter{footnote}{0}
\stepcounter{footnote}
  \footnotetext{\href{mailto:\andsunds}{\texttt{\andsunds}}}




%------------------------------------------------------------------------------------
%	DATE SECTION
%------------------------------------------------------------------------------------
% Följer ISO-standarden för tidsintervall:
% https://en.wikipedia.org/wiki/ISO_8601#Time_intervals
% "Double hyphen" också ok istället för '/'. -- i LaTeX är dock lite på gränsen
{ \large
\begin{tabular}{rc}
\today
\end{tabular}\\[1cm]
}

%------------------------------------------------------------------------------------
%	LOGO SECTION
%------------------------------------------------------------------------------------

\includegraphics[height=5cm]{logo.pdf} % Include a department/university logo
 
%------------------------------------------------------------------------------------

\vfill % Fill the rest of the page with whitespace

\end{titlepage}
\restoregeometry


\setcounter{page}{2}%detta är ANDRA (2) sidan


\begin{abstract}

\end{abstract}

\begin{otherlanguage}{english}
\begin{abstract}

\end{abstract}
\end{otherlanguage}

\clearpage
\renewcommand{\contentsname}{Innehållsförteckning}
\tableofcontents

\clearpage
\pagenumbering{arabic}
\setcounter{page}{1}

\renewcommand{\thefootnote}{\arabic{footnote}}
\setcounter{footnote}{0}
